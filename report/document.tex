\documentclass[12pt,a4paper]{article}

% Packages for enhanced formatting and functionality
\usepackage[utf8]{inputenc}
\usepackage[T1]{fontenc}
\usepackage[english]{babel}
\usepackage{geometry}
\usepackage{amsmath}
\usepackage{amsfonts}
\usepackage{amssymb}
\usepackage{graphicx}
\usepackage{url}
\usepackage{hyperref}

% Bibliography package
\usepackage{natbib}
\bibliographystyle{plain}

% Page geometry
\geometry{margin=1in}

% Title page information
\title{IM1202 Project Document}
\author{Student Name}
\date{\today}

\begin{document}

\maketitle

\begin{abstract}
This is a sample LaTeX document created for the IM1202 project. It demonstrates the use of LaTeX for academic writing, including proper citation formatting and bibliography management. The document serves as a template for future academic papers and reports.
\end{abstract}

\tableofcontents
\newpage

\section{Introduction}

LaTeX is a powerful typesetting system that is widely used in academia for producing high-quality documents \cite{lamport1994latex}. This document demonstrates the basic structure of a LaTeX document with proper bibliography integration.

The advantages of using LaTeX include:
\begin{itemize}
\item Professional typesetting quality
\item Excellent handling of mathematical formulas
\item Automatic cross-referencing and bibliography management
\item Consistent formatting throughout the document
\end{itemize}

\section{Methodology}

In academic writing, it is crucial to properly cite sources and maintain a consistent bibliography format. LaTeX provides excellent tools for this purpose through packages like \texttt{natbib} and \texttt{biblatex} \cite{patashnik1988bibtex}.

The workflow for creating documents with LaTeX typically involves:
\begin{enumerate}
\item Writing the main document in LaTeX format
\item Creating a bibliography database (.bib file)
\item Compiling the document using appropriate LaTeX engines
\item Generating the final PDF output
\end{enumerate}

\section{Results and Discussion}

Modern document preparation systems have evolved significantly since the introduction of LaTeX \cite{knuth1986texbook}. The integration of version control systems like Git with LaTeX documents allows for collaborative writing and automatic document generation through continuous integration pipelines.

\subsection{Automated Document Generation}

GitHub Actions provides an excellent platform for automating the compilation of LaTeX documents \cite{github2019actions}. This approach ensures that documents are always up-to-date and can be generated consistently across different environments.

\section{Conclusion}

This document serves as a template for future LaTeX projects in the IM1202 course. The combination of LaTeX for document preparation and GitHub Actions for automated compilation provides a robust workflow for academic writing.

Future work may include exploring advanced LaTeX features such as:
\begin{itemize}
\item Complex mathematical typesetting
\item Advanced table and figure handling
\item Custom document classes
\item Integration with reference management tools
\end{itemize}

% Bibliography
\bibliography{references}

\end{document}
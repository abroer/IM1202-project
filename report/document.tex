\documentclass[12pt,a4paper]{article}

% Packages for enhanced formatting and functionality
\usepackage[utf8]{inputenc}
\usepackage[T1]{fontenc}
\usepackage[english]{babel}
\usepackage{geometry}
\usepackage{amsmath}
\usepackage{amsfonts}
\usepackage{amssymb}
\usepackage{graphicx}
\usepackage{url}
\usepackage{hyperref}

\usepackage{listings}
\usepackage{xcolor}

\lstset{
	language=Python,
	basicstyle=\ttfamily\small,
	numbers=left,                % Zet regelnummering aan
	numberstyle=\tiny\color{black}, % Stijl van de nummers
	stepnumber=1,                % Nummer elke regel
	numbersep=5pt,               % Afstand tussen nummer en code
	backgroundcolor=\color{lightgray!20}, % Achtergrondkleur
	showstringspaces=false,
	breaklines=true,
	frame=single,
	captionpos=b
}

% Bibliography package
\usepackage{natbib}
\bibliographystyle{plain}

% Page geometry
\geometry{margin=1in}

% Title page information
\title{IM1202 Knight Tour Project Report}
\author{Stijn de Preter (852726504) \\ Arjan Broer (850166428)}
\date{\today}

\begin{document}

\maketitle

\begin{abstract}
This document presents a comprehensive overview of the research conducted for the IM1202 project. It includes an introduction to the problem, methodology, related work, experiments, results, and conclusions. The focus of the project is on solving the Knight Tour problem using Answer Set Programming (ASP). Key findings and insights are discussed throughout the document.
\end{abstract}

\tableofcontents
\newpage

\section{Introduction}\label{sec:introduction}

The Knight’s Tour problem is a classic problem in computer science.
It involves moving a knight piece on a chessboard such that it visits every square exactly once.
The challenge lies in the unique movement capabilities of the knight, which moves in an "L" shape:
    two squares in one direction and then one square perpendicular, or one square in one direction and then two squares perpendicular.

Research questions that we discuss in this report include:
\begin{itemize}
    \item How can the Knight’s Tour problem be logically modeled so that it is solvable with Answer Set Programming (ASP)?
    \item Which different logical models can be used to formulate the problem?
    \item How does the size of the chessboard affect the complexity and solution methods?
    \item How can heuristics and multi-shot solving techniques be applied to improve the efficiency of solving the Knight’s Tour problem with Answer Set Programming (ASP)?
\end{itemize}

To limit the scope of the project and allow for comparisons, we will focus on finding five results for the problem on a standard board (8x8 squares).
We look at Open tours meaning: The knight visits every square exactly once, but the ending square is possibly not a knight's move away from the starting square.
If no performance difference is observed, the square can be expanded until a noticeable difference in performance is achieved.
The starting point will also be fixed at a corner of the chessboard.

\section{Methodology}\label{sec:methodology}

Describe the methods used to address the research questions, including any algorithms, tools, or frameworks employed.
There are several methods to solve this. The following two methods are available (sources still to be found/added):
Method 1: A sequence is used. This means that we search for the next square of a chessboard from a specific square.
Method 2: "Connected" fields are searched for for all squares. This searches for combinations of all the fields where they are connected.
Other strategies may also exist.

\section{Related Work}\label{sec:related-work}

Discuss previous research and literature relevant to the Knight Tour problem and Answer Set Programming (ASP). Cite sources appropriately using \cite{lamport1994latex}.

A more mathematical analysis of the knight's tour problem can be found in \cite{parberry1997efficient}.

In \cite{hearn2003complexity}, the complexity of generalized chess problems is discussed and imlemented in ASP.

\section{Experiments and Results}\label{sec:experiments-and-results}

Detail the experiments conducted, the data collected, and the results obtained. Use tables and figures as necessary to illustrate findings.

\subsection{Methods to solve the knight problem}

There are several methods to solve the knight problem. In this section two methods are discussed.

\subsubsection{Method 1}

This approach models the problem as a sequence of steps (or time points). The first step starts at position (x=1,y=1)(x = 1, y = 1)(x=1,y=1). For example, the second step might be at (x=2,y=3)(x = 2, y = 3)(x=2,y=3). Each such combination of coordinates and step number is called a visit: at step zzz, the knight occupies square (x,y)(x, y)(x,y). Finally, a constraint ensures that every square on the board is visited exactly once within a total number of steps equal to the number of squares.

%nog te bekijken of we dit toevoegen als bijlage:
\begin{lstlisting}[caption={Methode 1}]
	% Board size
	#const n = 8.
	
	% Board
	square(1..n, 1..n).
	
	% Knight moves
	move(X1,Y1,X2,Y2) :- square(X1,Y1), square(X2,Y2), 1 = |X1-X2|, 2 = |Y1-Y2|.
	move(X1,Y1,X2,Y2) :- square(X1,Y1), square(X2,Y2), 2 = |X1-X2|, 1 = |Y1-Y2|.
	
	% Steps
	step(1..n*n).
	
	% Initial position at step 1
	visit(1,1,1).
	
	% For each step > 1, the knight must move from the previous square to a new square via a legal knight move.
	% The choice rule ensures exactly one move is made at each step.
	{ visit(X2,Y2,Z2) : visit(X1,Y1,Z1), move(X1,Y1,X2,Y2), Z2 = Z1 + 1 } = 1 :- step(Z2), Z2 > 1.
	
	% Constraint: no square may be visited more than once.
	:- visit(X,Y,Z1), visit(X,Y,Z2), Z1 != Z2.
	
	
	#show visit/3.
\end{lstlisting}

\subsubsection{Method 2}

The second method is based on reachability. For each square on the board, a possible predecessor square is selected (except for the starting square). The final solution only includes configurations that satisfy two conditions:
- Every square must be reachable from the start (no isolated loops or disconnected regions, such as two squares pointing only to each other).
- Each square can have at most one outgoing edge, ensuring the structure forms a single continuous path rather than branching.

\subsection{Performance of the methods}

Both methods are asked to generate five answers. The first takes 31.880 seconds to do this, while the second method takes 0.342 seconds. To better test the performance of the second model, the second method is asked to provide 10 000 answers, which takes 3.688 seconds.

To better understand performance, we look at the statistics.

\subsubsection{Method 1 - Performance}

\subsubsection{Method 2 - Performance}




\section{Discussion}\label{sec:discussion}
Interpret the results in the context of the research questions. Discuss any limitations and potential implications of the findings.

\section{Conclusion}\label{sec:conclusion}

Summarize the key points of the document and suggest directions for future research.

% Bibliography
\bibliography{references}

\end{document}